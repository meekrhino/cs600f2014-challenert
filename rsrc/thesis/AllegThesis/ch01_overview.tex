%
% $Id: ch01_overview
%
%   *******************************************************************
%   * SEE THE MAIN FILE "AllegThesis.tex" FOR MORE INFORMATION.       *
%   *******************************************************************

\chapter{Introduction}\label{ch:intro} % we can refer to chapter by the label

The purpose of this sample thesis is to show various \LaTeX\ formatting
commands. Information about content should be obtained through consultation
with the thesis readers and examination of past senior theses.
There are no fixed rules governing the number of chapters or their titles---old 
senior theses and discussions with the thesis readers are the best resources
for making such decisions.

Usually, a chapter begins with a paragraph or two that serves as a
sort of mini-introduction to the chapter. This is followed 
by numbered sections created with ``\verb$\section{...}$'', 
``\verb$\subsection{...}$'', etc.

\section{Motivation} \label{sec:motivation}
The first paragraph in each section is not indented---that is a standard style
that is enforced by the \LaTeX\ program. It should never be necessary to
insert commands to force paragraph indentation.

The thesis should avoid the use of second-person pronouns such as ``you''
and ``your,'' as well as the use of imperative statements (implied
second-person) such as ``Look
at \ldots'' or ``Consider the following example \ldots''. 
First person (``I,'' ``my,'' etc.) is sometimes acceptable, 
but should be used sparingly. (This is an appropriate topic 
for discussion with the first reader of the project.) Contractions (``don't'',
``can't'', etc.) are considered informal and should be avoided. Common
abbreviations such as ``math'' for ``mathematics'' or ``comp'' for 
``senior comprehensive project''  are also informal and not appropriate
in the final thesis.

Figures illustrating complex or hard-to-describe concepts are essential.
All figures should be fully explained in the text. For example,
Figure \ref{latexprocess} shows the first step in processing a senior thesis.
The user files consist of the main file, {\tt AllegThesis.tex}, and zero or
more additional files ({\tt ch01.tex}, {\tt ch02.tex} in the figure). Not shown
in the figure are image files, the bibliography, and other included
components (the ``\ldots etc.\ldots'' on the left side of the figure). 
Typing the command {\tt pdflatex} with the main file name
produces a number of additional files, including an {\tt .aux} file (with
information such a label references and citations),
a table of contents, or {\tt .toc}, file, a printable {\tt .pdf} file, and
a number of others, depending on the document.

%   *******************************************************************
%   * FIGURES ARE PLACED ACCORDING TO A SET OF CONSTRAINTS THAT CAN   *
%   * BE MANIPULATED TO SOME DEGREE.                                  *
%   * A SEARCH FOR "controlling latex floats" TURNS UP A NUMBER OF    *
%   * SITES THAT HAVE USEFUL INFORMATION, FOR EXAMPLE:                *
%   *                                                                 *
%   * http://mintaka.sdsu.edu/GF/bibliog/latex/floats.html            *
%   * http://goo.gl/aC8E8Q                                            *
%   * http://robjhyndman.com/hyndsight/latex-floats/                  *
%   *******************************************************************

\begin{figure}[htbp]
\centering
\includegraphics[width=3.5in]{latexprocess.fig}
\caption{The first step in creating a thesis document}
\label{latexprocess}
\end{figure}

\section{Current State of the Art}\label{sec:stateofart}
There are multiple
ways to create {\it italicized}, {\bf bold-face}, {\tt fixed-width}, and
{\sf sans-serif} fonts, as well as combinations, e.g., \textit{\textbf{
bold italic}} or \textit{\textsf{italic sans-serif}}. The \LaTeX\ source
file for this chapter shows some of the ways to achieve these effects.
It is customary to use fixed-width font for program constructs, e.g.,
``In the Java code accompanying the figure, {\tt n} stands for the 
number of generations to be simulated by the {\tt evolvePopulation()} method.'' 
Variables in mathematical equations are normally rendered in 
italics, e.g., ``In the performance equation, {\it cpuTime} is 
the average time over fifty runs of the program.''

%It is the privilege of the thesis author (in consultation with the
%project supervisor and other readers) to decide on the best way to 
%organize the sections and chapters in the way that makes the most sense.
%If the introduction begins with a motivating
%anecdote, perhaps this is best followed by defining a few terms or mentioning
%some major results that the reader should be aware of right from the 
%beginning. But 
It is important to get as quickly as possible to 
a concise statement of the {\it thesis}---the main question 
addressed by the project. This might merit a separate
section of the chapter.

\section{Goals of the Project}\label{sec:goals}
This section could also be entitled ``Thesis'' or something similar. 

A formal thesis statement should be a 
%\emph{falsifiable} % COMMENT OUT THINGS THAT YOU MAY LATER WISH TO PUT BACK
statement about 
the goal achieved by the project.  For a purely scientific
project, this is the hypothesis being tested; it should be a
\emph{falsifiable} statement, i.e., one that can be disproven through
an appropriate experiment.
For an applied programming project, it is usually a statement about 
the feasibility and correctness of the approach used and the advantages it 
has over other approaches, using suitable metrics.  For a survey or study,
it is usually a statement regarding the need 
or usefulness of such a study, its intended audience, and so on.

% COMMENTED OUT NEXT FEW LINES TO SAVE SPACE; MAY PUT THEM BACK LATER
%Following the concise statement of the thesis, some of the details can be
%expanded.  
%It is appropriate to
%refer to some of the results in the introduction (which may 
%mean going back and adding them to the introduction once the
%research is completed). 
%A senior thesis, or any research paper, is not a mystery 
%novel---there is no need to keep the reader in suspense about what
%has been accomplished.

\section{Thesis Outline}\label{sec:outline}
The introductory chapter usually concludes with a ``road map'' of the upcoming
chapters, e.g., ``Chapter \ref{ch:relatedwork} reviews a number of past approaches
to the problem and summarizes their strengths and weaknesses. Chapter 
\ref{ch:method} outlines the method of approach used to establish the
results.''
